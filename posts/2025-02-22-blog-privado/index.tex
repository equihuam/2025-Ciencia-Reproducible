% Options for packages loaded elsewhere
\PassOptionsToPackage{unicode}{hyperref}
\PassOptionsToPackage{hyphens}{url}
\PassOptionsToPackage{dvipsnames,svgnames,x11names}{xcolor}
%
\documentclass[
  letterpaper,
  DIV=11,
  numbers=noendperiod]{scrartcl}

\usepackage{amsmath,amssymb}
\usepackage{iftex}
\ifPDFTeX
  \usepackage[T1]{fontenc}
  \usepackage[utf8]{inputenc}
  \usepackage{textcomp} % provide euro and other symbols
\else % if luatex or xetex
  \usepackage{unicode-math}
  \defaultfontfeatures{Scale=MatchLowercase}
  \defaultfontfeatures[\rmfamily]{Ligatures=TeX,Scale=1}
\fi
\usepackage{lmodern}
\ifPDFTeX\else  
    % xetex/luatex font selection
\fi
% Use upquote if available, for straight quotes in verbatim environments
\IfFileExists{upquote.sty}{\usepackage{upquote}}{}
\IfFileExists{microtype.sty}{% use microtype if available
  \usepackage[]{microtype}
  \UseMicrotypeSet[protrusion]{basicmath} % disable protrusion for tt fonts
}{}
\makeatletter
\@ifundefined{KOMAClassName}{% if non-KOMA class
  \IfFileExists{parskip.sty}{%
    \usepackage{parskip}
  }{% else
    \setlength{\parindent}{0pt}
    \setlength{\parskip}{6pt plus 2pt minus 1pt}}
}{% if KOMA class
  \KOMAoptions{parskip=half}}
\makeatother
\usepackage{xcolor}
\setlength{\emergencystretch}{3em} % prevent overfull lines
\setcounter{secnumdepth}{-\maxdimen} % remove section numbering
% Make \paragraph and \subparagraph free-standing
\makeatletter
\ifx\paragraph\undefined\else
  \let\oldparagraph\paragraph
  \renewcommand{\paragraph}{
    \@ifstar
      \xxxParagraphStar
      \xxxParagraphNoStar
  }
  \newcommand{\xxxParagraphStar}[1]{\oldparagraph*{#1}\mbox{}}
  \newcommand{\xxxParagraphNoStar}[1]{\oldparagraph{#1}\mbox{}}
\fi
\ifx\subparagraph\undefined\else
  \let\oldsubparagraph\subparagraph
  \renewcommand{\subparagraph}{
    \@ifstar
      \xxxSubParagraphStar
      \xxxSubParagraphNoStar
  }
  \newcommand{\xxxSubParagraphStar}[1]{\oldsubparagraph*{#1}\mbox{}}
  \newcommand{\xxxSubParagraphNoStar}[1]{\oldsubparagraph{#1}\mbox{}}
\fi
\makeatother

\usepackage{color}
\usepackage{fancyvrb}
\newcommand{\VerbBar}{|}
\newcommand{\VERB}{\Verb[commandchars=\\\{\}]}
\DefineVerbatimEnvironment{Highlighting}{Verbatim}{commandchars=\\\{\}}
% Add ',fontsize=\small' for more characters per line
\usepackage{framed}
\definecolor{shadecolor}{RGB}{241,243,245}
\newenvironment{Shaded}{\begin{snugshade}}{\end{snugshade}}
\newcommand{\AlertTok}[1]{\textcolor[rgb]{0.68,0.00,0.00}{#1}}
\newcommand{\AnnotationTok}[1]{\textcolor[rgb]{0.37,0.37,0.37}{#1}}
\newcommand{\AttributeTok}[1]{\textcolor[rgb]{0.40,0.45,0.13}{#1}}
\newcommand{\BaseNTok}[1]{\textcolor[rgb]{0.68,0.00,0.00}{#1}}
\newcommand{\BuiltInTok}[1]{\textcolor[rgb]{0.00,0.23,0.31}{#1}}
\newcommand{\CharTok}[1]{\textcolor[rgb]{0.13,0.47,0.30}{#1}}
\newcommand{\CommentTok}[1]{\textcolor[rgb]{0.37,0.37,0.37}{#1}}
\newcommand{\CommentVarTok}[1]{\textcolor[rgb]{0.37,0.37,0.37}{\textit{#1}}}
\newcommand{\ConstantTok}[1]{\textcolor[rgb]{0.56,0.35,0.01}{#1}}
\newcommand{\ControlFlowTok}[1]{\textcolor[rgb]{0.00,0.23,0.31}{\textbf{#1}}}
\newcommand{\DataTypeTok}[1]{\textcolor[rgb]{0.68,0.00,0.00}{#1}}
\newcommand{\DecValTok}[1]{\textcolor[rgb]{0.68,0.00,0.00}{#1}}
\newcommand{\DocumentationTok}[1]{\textcolor[rgb]{0.37,0.37,0.37}{\textit{#1}}}
\newcommand{\ErrorTok}[1]{\textcolor[rgb]{0.68,0.00,0.00}{#1}}
\newcommand{\ExtensionTok}[1]{\textcolor[rgb]{0.00,0.23,0.31}{#1}}
\newcommand{\FloatTok}[1]{\textcolor[rgb]{0.68,0.00,0.00}{#1}}
\newcommand{\FunctionTok}[1]{\textcolor[rgb]{0.28,0.35,0.67}{#1}}
\newcommand{\ImportTok}[1]{\textcolor[rgb]{0.00,0.46,0.62}{#1}}
\newcommand{\InformationTok}[1]{\textcolor[rgb]{0.37,0.37,0.37}{#1}}
\newcommand{\KeywordTok}[1]{\textcolor[rgb]{0.00,0.23,0.31}{\textbf{#1}}}
\newcommand{\NormalTok}[1]{\textcolor[rgb]{0.00,0.23,0.31}{#1}}
\newcommand{\OperatorTok}[1]{\textcolor[rgb]{0.37,0.37,0.37}{#1}}
\newcommand{\OtherTok}[1]{\textcolor[rgb]{0.00,0.23,0.31}{#1}}
\newcommand{\PreprocessorTok}[1]{\textcolor[rgb]{0.68,0.00,0.00}{#1}}
\newcommand{\RegionMarkerTok}[1]{\textcolor[rgb]{0.00,0.23,0.31}{#1}}
\newcommand{\SpecialCharTok}[1]{\textcolor[rgb]{0.37,0.37,0.37}{#1}}
\newcommand{\SpecialStringTok}[1]{\textcolor[rgb]{0.13,0.47,0.30}{#1}}
\newcommand{\StringTok}[1]{\textcolor[rgb]{0.13,0.47,0.30}{#1}}
\newcommand{\VariableTok}[1]{\textcolor[rgb]{0.07,0.07,0.07}{#1}}
\newcommand{\VerbatimStringTok}[1]{\textcolor[rgb]{0.13,0.47,0.30}{#1}}
\newcommand{\WarningTok}[1]{\textcolor[rgb]{0.37,0.37,0.37}{\textit{#1}}}

\providecommand{\tightlist}{%
  \setlength{\itemsep}{0pt}\setlength{\parskip}{0pt}}\usepackage{longtable,booktabs,array}
\usepackage{calc} % for calculating minipage widths
% Correct order of tables after \paragraph or \subparagraph
\usepackage{etoolbox}
\makeatletter
\patchcmd\longtable{\par}{\if@noskipsec\mbox{}\fi\par}{}{}
\makeatother
% Allow footnotes in longtable head/foot
\IfFileExists{footnotehyper.sty}{\usepackage{footnotehyper}}{\usepackage{footnote}}
\makesavenoteenv{longtable}
\usepackage{graphicx}
\makeatletter
\newsavebox\pandoc@box
\newcommand*\pandocbounded[1]{% scales image to fit in text height/width
  \sbox\pandoc@box{#1}%
  \Gscale@div\@tempa{\textheight}{\dimexpr\ht\pandoc@box+\dp\pandoc@box\relax}%
  \Gscale@div\@tempb{\linewidth}{\wd\pandoc@box}%
  \ifdim\@tempb\p@<\@tempa\p@\let\@tempa\@tempb\fi% select the smaller of both
  \ifdim\@tempa\p@<\p@\scalebox{\@tempa}{\usebox\pandoc@box}%
  \else\usebox{\pandoc@box}%
  \fi%
}
% Set default figure placement to htbp
\def\fps@figure{htbp}
\makeatother

<meta name="robots" content="noindex">
\KOMAoption{captions}{tableheading}
\makeatletter
\@ifpackageloaded{tcolorbox}{}{\usepackage[skins,breakable]{tcolorbox}}
\@ifpackageloaded{fontawesome5}{}{\usepackage{fontawesome5}}
\definecolor{quarto-callout-color}{HTML}{909090}
\definecolor{quarto-callout-note-color}{HTML}{0758E5}
\definecolor{quarto-callout-important-color}{HTML}{CC1914}
\definecolor{quarto-callout-warning-color}{HTML}{EB9113}
\definecolor{quarto-callout-tip-color}{HTML}{00A047}
\definecolor{quarto-callout-caution-color}{HTML}{FC5300}
\definecolor{quarto-callout-color-frame}{HTML}{acacac}
\definecolor{quarto-callout-note-color-frame}{HTML}{4582ec}
\definecolor{quarto-callout-important-color-frame}{HTML}{d9534f}
\definecolor{quarto-callout-warning-color-frame}{HTML}{f0ad4e}
\definecolor{quarto-callout-tip-color-frame}{HTML}{02b875}
\definecolor{quarto-callout-caution-color-frame}{HTML}{fd7e14}
\makeatother
\makeatletter
\@ifpackageloaded{caption}{}{\usepackage{caption}}
\AtBeginDocument{%
\ifdefined\contentsname
  \renewcommand*\contentsname{Tabla de contenidos}
\else
  \newcommand\contentsname{Tabla de contenidos}
\fi
\ifdefined\listfigurename
  \renewcommand*\listfigurename{Listado de Figuras}
\else
  \newcommand\listfigurename{Listado de Figuras}
\fi
\ifdefined\listtablename
  \renewcommand*\listtablename{Listado de Tablas}
\else
  \newcommand\listtablename{Listado de Tablas}
\fi
\ifdefined\figurename
  \renewcommand*\figurename{Figura}
\else
  \newcommand\figurename{Figura}
\fi
\ifdefined\tablename
  \renewcommand*\tablename{Tabla}
\else
  \newcommand\tablename{Tabla}
\fi
}
\@ifpackageloaded{float}{}{\usepackage{float}}
\floatstyle{ruled}
\@ifundefined{c@chapter}{\newfloat{codelisting}{h}{lop}}{\newfloat{codelisting}{h}{lop}[chapter]}
\floatname{codelisting}{Listado}
\newcommand*\listoflistings{\listof{codelisting}{Listado de Listados}}
\makeatother
\makeatletter
\makeatother
\makeatletter
\@ifpackageloaded{caption}{}{\usepackage{caption}}
\@ifpackageloaded{subcaption}{}{\usepackage{subcaption}}
\makeatother
\makeatletter
\@ifpackageloaded{tikz}{}{\usepackage{tikz}}
\makeatother
        \newcommand*\circled[1]{\tikz[baseline=(char.base)]{
          \node[shape=circle,draw,inner sep=1pt] (char) {{\scriptsize#1}};}}  
                  

\ifLuaTeX
\usepackage[bidi=basic]{babel}
\else
\usepackage[bidi=default]{babel}
\fi
\babelprovide[main,import]{spanish}
% get rid of language-specific shorthands (see #6817):
\let\LanguageShortHands\languageshorthands
\def\languageshorthands#1{}
\usepackage{bookmark}

\IfFileExists{xurl.sty}{\usepackage{xurl}}{} % add URL line breaks if available
\urlstyle{same} % disable monospaced font for URLs
\hypersetup{
  pdftitle={Hacer que mi blog sea privado},
  pdfauthor={Miguel},
  pdflang={es},
  colorlinks=true,
  linkcolor={blue},
  filecolor={Maroon},
  citecolor={Blue},
  urlcolor={Blue},
  pdfcreator={LaTeX via pandoc}}


\title{Hacer que mi blog sea privado}
\author{Miguel}
\date{}

\begin{document}
\maketitle


Supongamos que quieres hacer que sólo los coautores del trabajo tengan
acceso al \emph{blog} o a alguna sección del mismo. Esto puede iniciar
procurando que los \emph{bots} o \emph{arañas} de \emph{Google} y otros
buscadores no recojan datos o metan las narices en mi contenido. Bueno,
esto puede hacerse por dos caminos. Lo primero lo haremos con el
\href{https://seranking.com/es/blog/noindex/}{concepto \textbf{noindex}}
y la ayuda de instrucciones directas a los \emph{bots} mediante un
archivo de indicaciones. El resultado es poco invasivo para el usuario
pero no enteramente seguro para garantizar la privacidad del contenido.
Una solución más radical es de plano encriptar el sitio o porciones
sensibles del mismo, para sólo mostrarlo cuando se proporciones una
clave específica. Un archivo encriptado no es indexable (es parte de la
\emph{Deep web}, que no es lo mismo que la \emph{Dark Web}). Este último
método es muy efectivo, pero claro, potencialmente un poco incómodo para
el usuario. Desde luego, las fugas de información son posibles, pero más
bien vinculadas al cuidado que tengan los involucrados en la gestión del
acceso y del manejo del contenido. Si todo esto es necesario en un marco
de \emph{ciencia abierta} es algo que vale la pena meditar, pues el
acceso temprano es congruente con las nociones de \emph{pre-registro} y
\emph{pre-publicación}.

\subsubsection{No index}\label{no-index}

Lo que hay que hacer para evitar a los \emph{bots} en la página
principal, es poner en el archivo \texttt{\textbackslash{}\_quarto.yml}
las siguientes líneas:

\begin{tcolorbox}[enhanced jigsaw, toprule=.15mm, left=2mm, opacityback=0, colback=white, titlerule=0mm, title=\textcolor{quarto-callout-tip-color}{\faLightbulb}\hspace{0.5em}{Archivo \texttt{\_quarto.yml}}, toptitle=1mm, breakable, arc=.35mm, rightrule=.15mm, coltitle=black, colframe=quarto-callout-tip-color-frame, bottomtitle=1mm, bottomrule=.15mm, leftrule=.75mm, colbacktitle=quarto-callout-tip-color!10!white, opacitybacktitle=0.6]

\phantomsection\label{annotated-cell-1}%
\begin{Shaded}
\begin{Highlighting}[]

\FunctionTok{project}\KeywordTok{:}
\AttributeTok{  }\FunctionTok{type}\KeywordTok{:}\AttributeTok{ website}
\AttributeTok{  }\FunctionTok{resources}\KeywordTok{:}\CommentTok{ }\hspace*{\fill}\NormalTok{\circled{1}}
\AttributeTok{    }\KeywordTok{{-}}\AttributeTok{ robots.txt}
\AttributeTok{    }
\FunctionTok{format}\KeywordTok{:}
\AttributeTok{  }\FunctionTok{html}\KeywordTok{:}
\AttributeTok{    }\FunctionTok{theme}\KeywordTok{:}\AttributeTok{ fotoD.png}
\AttributeTok{    }\FunctionTok{include{-}in{-}header}\KeywordTok{:}\CommentTok{ }\hspace*{\fill}\NormalTok{\circled{2}}
\AttributeTok{      }\KeywordTok{{-}}\AttributeTok{ }\FunctionTok{text}\KeywordTok{:}\AttributeTok{ \textless{}meta name="robots" content="noindex"\textgreater{}}
\end{Highlighting}
\end{Shaded}

\begin{description}
\tightlist
\item[\circled{1}]
Hay que proporcionar una lista de instrucciones para los robots que
andan cosechando por la Web. Se hace mediante un archivo específico como
el anotado.
\item[\circled{2}]
Esta es tu petición de no indexar por favor. Sigue el código concreto
que lo indica, que debe aparecer en la sección \texttt{head} del sitio
Web que contiene tu \emph{Blog}.
\end{description}

\end{tcolorbox}

Con esto evitaras que muchos buscadores, los más formales, te den gusto
y no reúnan información de tu sitio. No todos los buscadores cumplen con
este estándar, pero sí lo hacen los más importantes. Esto también te
hace ver que es un \emph{acto voluntario}, es decir, un metiche puede
optar por ignorar tus deseos.

Las instrucciones para los robots pueden variar. En este caso puse las
siguientes en el archivo \texttt{robots.txt} en la raíz de tu proyecto.

\begin{tcolorbox}[enhanced jigsaw, toprule=.15mm, left=2mm, opacityback=0, colback=white, titlerule=0mm, title=\textcolor{quarto-callout-tip-color}{\faLightbulb}\hspace{0.5em}{Archivo robots.txt}, toptitle=1mm, breakable, arc=.35mm, rightrule=.15mm, coltitle=black, colframe=quarto-callout-tip-color-frame, bottomtitle=1mm, bottomrule=.15mm, leftrule=.75mm, colbacktitle=quarto-callout-tip-color!10!white, opacitybacktitle=0.6]

\begin{Shaded}
\begin{Highlighting}[]
\NormalTok{robots.txt generated by www.seoptimer.com}
\NormalTok{User{-}agent: Googlebot}
\NormalTok{Disallow: /}
\NormalTok{User{-}agent: googlebot{-}image}
\NormalTok{Disallow: /}
\NormalTok{User{-}agent: googlebot{-}mobile}
\NormalTok{Disallow: /}
\NormalTok{User{-}agent: MSNBot}
\NormalTok{Disallow: /}
\NormalTok{User{-}agent: Slurp}
\NormalTok{Disallow: /}
\NormalTok{User{-}agent: Teoma}
\NormalTok{Disallow: /}
\NormalTok{User{-}agent: Gigabot}
\NormalTok{Disallow: /}
\NormalTok{User{-}agent: Robozilla}
\NormalTok{Disallow: /}
\NormalTok{User{-}agent: Nutch}
\NormalTok{Disallow: /}
\NormalTok{User{-}agent: ia\_archiver}
\NormalTok{Disallow: /}
\NormalTok{User{-}agent: baiduspider}
\NormalTok{Disallow: /}
\NormalTok{User{-}agent: naverbot}
\NormalTok{Disallow: /}
\NormalTok{User{-}agent: yeti}
\NormalTok{Disallow: /}
\NormalTok{User{-}agent: yahoo{-}mmcrawler}
\NormalTok{Disallow: /}
\NormalTok{User{-}agent: psbot}
\NormalTok{Disallow: /}
\NormalTok{User{-}agent: yahoo{-}blogs/v3.9}
\NormalTok{Disallow: /}
\NormalTok{User{-}agent: *}
\NormalTok{Disallow: /}
\NormalTok{Disallow: /cgi{-}bin/}
\end{Highlighting}
\end{Shaded}

\end{tcolorbox}

~

Una solución más definitiva es encriptar tu contenido. En este caso te
sugiero usar la solución que se ofrece con la biblioteca
\texttt{staticryptR}. Las
\href{https://cran.r-project.org/web/packages/staticryptR/readme/README.html}{explicaciones
de cómo y qué instalar están acá}. Si la tradicional instalación de
librerías no es suficiente, te sugiero ir al documento sugerido y ver
las líneas que hablan de \emph{Node.js}, un ambiente de
\emph{javascript} que agrega una capa de procesamiento necesario para
esto en tu máquina. Luego, asegúrate de tener instalada en \textbf{R} la
biblioteca mencionada. Con eso hecho. Ve de nuevo a tu archivo
\texttt{\textbackslash{}\_quarto.yml} y agrega las líneas necesarias
para que incluya lo que se indica:

\begin{tcolorbox}[enhanced jigsaw, toprule=.15mm, left=2mm, opacityback=0, colback=white, titlerule=0mm, title=\textcolor{quarto-callout-tip-color}{\faLightbulb}\hspace{0.5em}{Orden de encriptar en \texttt{\textbackslash{}\_quarto.yml}}, toptitle=1mm, breakable, arc=.35mm, rightrule=.15mm, coltitle=black, colframe=quarto-callout-tip-color-frame, bottomtitle=1mm, bottomrule=.15mm, leftrule=.75mm, colbacktitle=quarto-callout-tip-color!10!white, opacitybacktitle=0.6]

\phantomsection\label{annotated-cell-3}%
\begin{Shaded}
\begin{Highlighting}[]
\FunctionTok{project}\KeywordTok{:}
\AttributeTok{  }\FunctionTok{type}\KeywordTok{:}\AttributeTok{ website}
\AttributeTok{  }\FunctionTok{output{-}dir}\KeywordTok{:}\AttributeTok{ }\StringTok{"./\_site"}\CommentTok{ }\hspace*{\fill}\NormalTok{\circled{1}}
\AttributeTok{  }\FunctionTok{post{-}render}\KeywordTok{:}\AttributeTok{ encrypt.r}\CommentTok{ }\hspace*{\fill}\NormalTok{\circled{2}}
\AttributeTok{  }\FunctionTok{resources}\KeywordTok{:}\AttributeTok{ }
\AttributeTok{    }\KeywordTok{{-}}\AttributeTok{ robots.txt}
\end{Highlighting}
\end{Shaded}

\begin{description}
\tightlist
\item[\circled{1}]
Indica el archivo o directorio que quieres encriptar, en este caso es
\textbf{todo el sitio}, qué vive en el directorio
\texttt{\textbackslash{}\_site}.
\item[\circled{2}]
Anota el archivo que contiene las instrucciones concretas sobre qué y
cómo encriptar el contenido.
\end{description}

\end{tcolorbox}

Desde luego, también necesitas el código indicado que pondrás en el
archivo \texttt{encrypt.r} y que alojaras en la raíz de tu proyecto.

\begin{tcolorbox}[enhanced jigsaw, toprule=.15mm, left=2mm, opacityback=0, colback=white, titlerule=0mm, title=\textcolor{quarto-callout-tip-color}{\faLightbulb}\hspace{0.5em}{Código para crear el archivo encrypt.r}, toptitle=1mm, breakable, arc=.35mm, rightrule=.15mm, coltitle=black, colframe=quarto-callout-tip-color-frame, bottomtitle=1mm, bottomrule=.15mm, leftrule=.75mm, colbacktitle=quarto-callout-tip-color!10!white, opacitybacktitle=0.6]

\paragraph{\texorpdfstring{archivo
\texttt{encrypt.r}}{archivo encrypt.r}}\label{archivo-encrypt.r}

\phantomsection\label{annotated-cell-4}%
\begin{Shaded}
\begin{Highlighting}[]
\NormalTok{staticryptR}\SpecialCharTok{::}\FunctionTok{staticryptr}\NormalTok{(}
  \AttributeTok{files =} \StringTok{"\_site/"}\NormalTok{, }\hspace*{\fill}\NormalTok{\circled{1}}
  \AttributeTok{directory =} \StringTok{"."}\NormalTok{, }\hspace*{\fill}\NormalTok{\circled{2}}
  \AttributeTok{password =} \StringTok{"pita{-}123"}\NormalTok{, }\hspace*{\fill}\NormalTok{\circled{3}}
  \AttributeTok{short =} \ConstantTok{TRUE}\NormalTok{, }\hspace*{\fill}\NormalTok{\circled{4}}
  \AttributeTok{recursive =} \ConstantTok{TRUE} \hspace*{\fill}\NormalTok{\circled{5}}
\NormalTok{)}
\end{Highlighting}
\end{Shaded}

\begin{description}
\tightlist
\item[\circled{1}]
Directorio objetivo, en este caso: todo el sitio.
\item[\circled{2}]
En dónde encontrar el directorio objetivo. En este caso la raíz de tu
proyecto.
\item[\circled{3}]
La clave que usaras para la tarea (sólo una), puedes usar
\texttt{keyring} para ocultarla.
\item[\circled{4}]
Indicas que es válido usar claves cortas (quizás sólo 8 caracteres). Es
preferible claves largas.
\item[\circled{5}]
Especifica que quieres encriptar la carpeta indicada y todas las
subcarpetas que contenga.
\end{description}

\end{tcolorbox}

\subsection{Cuidado}\label{cuidado}

Al encriptar \texttt{\_site} los paquetes ya codificados se van
``acumulando'', lo que se traduce en que tengan un tamaño cada vez
mayor, hasta volverse inmanejables. Para evitar ese efecto no deseado,
es conveniente usar preferentemente la opción \textbf{Render Website}
desde la pestaña \textbf{Build} en \emph{RStudio}. Esto recrea todos los
archivos desde cero. Archivos muy grandes acaban siendo un obstáculo
para el manejo eficiente en \emph{git}. Para ayudar a visualizar el
problema agregué unas lineas en el escript de \texttt{encrypt.r} que te
ayudarán a visualizar si están apareciendo archivos potencialmente
problemáticos, es decir mayores a 10MB. La advertencia la podrás ver en
la pestaña \emph{Background Jobs}, pero sólo aparecerá si se detecta
algún riesgo.

\begin{tcolorbox}[enhanced jigsaw, toprule=.15mm, left=2mm, opacityback=0, colback=white, titlerule=0mm, title=\textcolor{quarto-callout-tip-color}{\faLightbulb}\hspace{0.5em}{Vigilar tamaño de archivos en \_site}, toptitle=1mm, breakable, arc=.35mm, rightrule=.15mm, coltitle=black, colframe=quarto-callout-tip-color-frame, bottomtitle=1mm, bottomrule=.15mm, leftrule=.75mm, colbacktitle=quarto-callout-tip-color!10!white, opacitybacktitle=0.6]

Sí te interesa mantener la vigilancia del tamaño de los archivos en
\texttt{\textbackslash{}\_site}, este es un código que lo hace y te
reporta cuando se sospecha de algún riesgo por archivos mayores a 10MB.

\begin{Shaded}
\begin{Highlighting}[]
\NormalTok{lista\_rchivos }\OtherTok{\textless{}{-}}\NormalTok{ tibble}\SpecialCharTok{::}\FunctionTok{tibble}\NormalTok{(}\AttributeTok{archivo =} \FunctionTok{list.files}\NormalTok{(}\StringTok{"./\_site/"}\NormalTok{, }
                                                     \AttributeTok{recursive =} \ConstantTok{TRUE}\NormalTok{,}
                                                     \AttributeTok{full.names =} \ConstantTok{TRUE}\NormalTok{)) }\SpecialCharTok{|\textgreater{}} 
\NormalTok{  dplyr}\SpecialCharTok{::}\FunctionTok{mutate}\NormalTok{(}\AttributeTok{MB =} \FunctionTok{file.size}\NormalTok{(archivo)}\SpecialCharTok{/}\NormalTok{(}\DecValTok{1024}\SpecialCharTok{\^{}}\DecValTok{2}\NormalTok{)) }\SpecialCharTok{|\textgreater{}} 
\NormalTok{  dplyr}\SpecialCharTok{::}\FunctionTok{filter}\NormalTok{(MB }\SpecialCharTok{\textgreater{}} \DecValTok{10}\NormalTok{)}
                                  
\ControlFlowTok{if}\NormalTok{ (}\FunctionTok{dim}\NormalTok{(lista\_rchivos)[}\DecValTok{1}\NormalTok{] }\SpecialCharTok{\textgreater{}} \DecValTok{0}\NormalTok{)}
\NormalTok{\{}
  \FunctionTok{print}\NormalTok{(}\StringTok{"Advertencia: hay archivos muy grandes en \_site"}\NormalTok{)}
  \FunctionTok{print}\NormalTok{(}\StringTok{"Te sugiero hacer un render completo desde la pestaña \textquotesingle{}Build\textquotesingle{}"}\NormalTok{)}
  \FunctionTok{print}\NormalTok{(lista\_rchivos)}
\NormalTok{\}}
\end{Highlighting}
\end{Shaded}

\end{tcolorbox}




\end{document}
