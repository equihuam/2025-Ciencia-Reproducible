% Options for packages loaded elsewhere
\PassOptionsToPackage{unicode}{hyperref}
\PassOptionsToPackage{hyphens}{url}
\PassOptionsToPackage{dvipsnames,svgnames,x11names}{xcolor}
%
\documentclass[
  letterpaper,
  DIV=11,
  numbers=noendperiod]{scrartcl}

\usepackage{amsmath,amssymb}
\usepackage{iftex}
\ifPDFTeX
  \usepackage[T1]{fontenc}
  \usepackage[utf8]{inputenc}
  \usepackage{textcomp} % provide euro and other symbols
\else % if luatex or xetex
  \usepackage{unicode-math}
  \defaultfontfeatures{Scale=MatchLowercase}
  \defaultfontfeatures[\rmfamily]{Ligatures=TeX,Scale=1}
\fi
\usepackage{lmodern}
\ifPDFTeX\else  
    % xetex/luatex font selection
\fi
% Use upquote if available, for straight quotes in verbatim environments
\IfFileExists{upquote.sty}{\usepackage{upquote}}{}
\IfFileExists{microtype.sty}{% use microtype if available
  \usepackage[]{microtype}
  \UseMicrotypeSet[protrusion]{basicmath} % disable protrusion for tt fonts
}{}
\makeatletter
\@ifundefined{KOMAClassName}{% if non-KOMA class
  \IfFileExists{parskip.sty}{%
    \usepackage{parskip}
  }{% else
    \setlength{\parindent}{0pt}
    \setlength{\parskip}{6pt plus 2pt minus 1pt}}
}{% if KOMA class
  \KOMAoptions{parskip=half}}
\makeatother
\usepackage{xcolor}
\setlength{\emergencystretch}{3em} % prevent overfull lines
\setcounter{secnumdepth}{-\maxdimen} % remove section numbering
% Make \paragraph and \subparagraph free-standing
\makeatletter
\ifx\paragraph\undefined\else
  \let\oldparagraph\paragraph
  \renewcommand{\paragraph}{
    \@ifstar
      \xxxParagraphStar
      \xxxParagraphNoStar
  }
  \newcommand{\xxxParagraphStar}[1]{\oldparagraph*{#1}\mbox{}}
  \newcommand{\xxxParagraphNoStar}[1]{\oldparagraph{#1}\mbox{}}
\fi
\ifx\subparagraph\undefined\else
  \let\oldsubparagraph\subparagraph
  \renewcommand{\subparagraph}{
    \@ifstar
      \xxxSubParagraphStar
      \xxxSubParagraphNoStar
  }
  \newcommand{\xxxSubParagraphStar}[1]{\oldsubparagraph*{#1}\mbox{}}
  \newcommand{\xxxSubParagraphNoStar}[1]{\oldsubparagraph{#1}\mbox{}}
\fi
\makeatother


\providecommand{\tightlist}{%
  \setlength{\itemsep}{0pt}\setlength{\parskip}{0pt}}\usepackage{longtable,booktabs,array}
\usepackage{calc} % for calculating minipage widths
% Correct order of tables after \paragraph or \subparagraph
\usepackage{etoolbox}
\makeatletter
\patchcmd\longtable{\par}{\if@noskipsec\mbox{}\fi\par}{}{}
\makeatother
% Allow footnotes in longtable head/foot
\IfFileExists{footnotehyper.sty}{\usepackage{footnotehyper}}{\usepackage{footnote}}
\makesavenoteenv{longtable}
\usepackage{graphicx}
\makeatletter
\def\maxwidth{\ifdim\Gin@nat@width>\linewidth\linewidth\else\Gin@nat@width\fi}
\def\maxheight{\ifdim\Gin@nat@height>\textheight\textheight\else\Gin@nat@height\fi}
\makeatother
% Scale images if necessary, so that they will not overflow the page
% margins by default, and it is still possible to overwrite the defaults
% using explicit options in \includegraphics[width, height, ...]{}
\setkeys{Gin}{width=\maxwidth,height=\maxheight,keepaspectratio}
% Set default figure placement to htbp
\makeatletter
\def\fps@figure{htbp}
\makeatother
% definitions for citeproc citations
\NewDocumentCommand\citeproctext{}{}
\NewDocumentCommand\citeproc{mm}{%
  \begingroup\def\citeproctext{#2}\cite{#1}\endgroup}
\makeatletter
 % allow citations to break across lines
 \let\@cite@ofmt\@firstofone
 % avoid brackets around text for \cite:
 \def\@biblabel#1{}
 \def\@cite#1#2{{#1\if@tempswa , #2\fi}}
\makeatother
\newlength{\cslhangindent}
\setlength{\cslhangindent}{1.5em}
\newlength{\csllabelwidth}
\setlength{\csllabelwidth}{3em}
\newenvironment{CSLReferences}[2] % #1 hanging-indent, #2 entry-spacing
 {\begin{list}{}{%
  \setlength{\itemindent}{0pt}
  \setlength{\leftmargin}{0pt}
  \setlength{\parsep}{0pt}
  % turn on hanging indent if param 1 is 1
  \ifodd #1
   \setlength{\leftmargin}{\cslhangindent}
   \setlength{\itemindent}{-1\cslhangindent}
  \fi
  % set entry spacing
  \setlength{\itemsep}{#2\baselineskip}}}
 {\end{list}}
\usepackage{calc}
\newcommand{\CSLBlock}[1]{\hfill\break\parbox[t]{\linewidth}{\strut\ignorespaces#1\strut}}
\newcommand{\CSLLeftMargin}[1]{\parbox[t]{\csllabelwidth}{\strut#1\strut}}
\newcommand{\CSLRightInline}[1]{\parbox[t]{\linewidth - \csllabelwidth}{\strut#1\strut}}
\newcommand{\CSLIndent}[1]{\hspace{\cslhangindent}#1}

\KOMAoption{captions}{tableheading,figureheading}
\makeatletter
\@ifpackageloaded{caption}{}{\usepackage{caption}}
\AtBeginDocument{%
\ifdefined\contentsname
  \renewcommand*\contentsname{Tabla de contenidos}
\else
  \newcommand\contentsname{Tabla de contenidos}
\fi
\ifdefined\listfigurename
  \renewcommand*\listfigurename{Listado de Figuras}
\else
  \newcommand\listfigurename{Listado de Figuras}
\fi
\ifdefined\listtablename
  \renewcommand*\listtablename{Lista de Cuadros}
\else
  \newcommand\listtablename{Lista de Cuadros}
\fi
\ifdefined\figurename
  \renewcommand*\figurename{Figura}
\else
  \newcommand\figurename{Figura}
\fi
\ifdefined\tablename
  \renewcommand*\tablename{Cuadro}
\else
  \newcommand\tablename{Cuadro}
\fi
}
\@ifpackageloaded{float}{}{\usepackage{float}}
\floatstyle{ruled}
\@ifundefined{c@chapter}{\newfloat{codelisting}{h}{lop}}{\newfloat{codelisting}{h}{lop}[chapter]}
\floatname{codelisting}{Listado}
\newcommand*\listoflistings{\listof{codelisting}{Listado de Listados}}
\makeatother
\makeatletter
\makeatother
\makeatletter
\@ifpackageloaded{caption}{}{\usepackage{caption}}
\@ifpackageloaded{subcaption}{}{\usepackage{subcaption}}
\makeatother

\ifLuaTeX
\usepackage[bidi=basic]{babel}
\else
\usepackage[bidi=default]{babel}
\fi
\babelprovide[main,import]{spanish}
% get rid of language-specific shorthands (see #6817):
\let\LanguageShortHands\languageshorthands
\def\languageshorthands#1{}
\ifLuaTeX
  \usepackage{selnolig}  % disable illegal ligatures
\fi
\usepackage{bookmark}

\IfFileExists{xurl.sty}{\usepackage{xurl}}{} % add URL line breaks if available
\urlstyle{same} % disable monospaced font for URLs
\hypersetup{
  pdftitle={¿Qué tipo de proyecto presenta mi blog?},
  pdfauthor={Miguel Equihua},
  pdflang={es},
  colorlinks=true,
  linkcolor={blue},
  filecolor={Maroon},
  citecolor={Blue},
  urlcolor={Blue},
  pdfcreator={LaTeX via pandoc}}


\title{¿Qué tipo de proyecto presenta mi blog?}
\author{Miguel Equihua}
\date{Xalapa, Ver., 21 de marzo, 2025}

\begin{document}
\maketitle


Los proyectos que suelen abordarse en una especialización científca o
tecnológica abordan desafíos de \textbf{identificación y medición de
relaciones causales}, estimación del estado de variables ambientales,
demográficas, económicas, etc., representación gráfica de procesos,
mecanismos o distribución geográfica de variables de interés. También
puede interesar la construcción de formas de comunicación efectiva a
actores relevantes pero no especializados en términos de preparación
universitaria formal. Será tu desafío precisar la naturaleza del
proyecto que será la base formal de tu \textbf{Blog}. Recuerda mantener
una perspectiva de \emph{narrativa basada en datos}, \emph{trazabilidad}
de la evidencia y de los procesos analíticos utilizados.

Para ayudarte a poner tus intereses en perspectiva conviene hablar sobre
como se caractefrizan generalmente los tipos de estudio científicos y
tecnológicos. Un pri mer asunto es considerar si se trata de estudios
que se basaran en datos ya existentes (\emph{retrospectivao}) o datos
nuevos que se planea producir (\emph{prospectivo}).

Otro aspecto a considerar es si para los fines del trabajo se requiere
una sóla medición en el tiempo (\emph{transversal}) o si se requiere dar
sguimiento a la evolución del fenómeno con mediciones repetidas varias
veces a lo largo del tiempo (\emph{longitudinal}).

Finalmente es necesario resolver si los propósitos del estudio se
requiere caracterizar una sóla población objetivo (\emph{descriptivo}) o
si interesa comparar y contrastar dos o más poblaciones
(\emph{comparativo}). Desde luego por \textbf{población} me refiero a la
idea estadística que propone una definición abstracta de un conjunto de
entidades observables, que se caracterizan con referencia precisa a un
conjunto de atributos o variables concretas.

A partir de estos elementos es posible clasificar los tipos de estudios
que se realizan. Desafortunadamente, las distintas disciplinas tienden a
denominar cada combinación de atributos de maneras algo peculiares. No
me ocuparé de eso, lo que interesa aquí es ayudarte a clarificar la
naturaleza de tu proyecto. Usualmente un estudio observacional, y
descriptivo y retrospectivo corresponde con la práctica de
\emph{revisión de casos} o de análisis de \emph{microdatos censales}.
Esto también se puede hacer con un propósito comparativo, los médicos y
epidemiólogos lo llaman estudio de \emph{casos y controles}, o
\emph{perspectiva histórica} tú ¿cómo lo llamarías?. Finalmente, si el
planteamiento es experimental y prospectivo, estás en el ámbito de un
\emph{experimento controlado}.

En todos estos casos existe una gran oferta de métodos estadísticos de
procesamiento de datos para evitar el \emph{efecto distractor} de los
factores y procesos de confusión que dificultan comprender la
consistencia y relevancia de los hallazgos científicos y tecnológicos.
Recientemente se ha avanzado mucho a este respecto con el uso de
diagramas causales y la comorensión de los patrones de correlación que
cabría esperar en caso de que sean razonablemente apegados al proceso
que estemos analizando. Estos diagramas son conocidos como \textbf{DAG}
(\emph{Directed Acyclic Graphs}) y encontraras una referencia
interesante a ellos en (Pearl \& Mackenzie, 2020).

Si tu estudio es más de tipo retrospectivo es decir buscas analizar
datos estadísticos o vas a realizr una encuesta, el libro en línea de
(Velásquez, s.~f.) es una fuente iteresante. Si por el contrario, tef
inclinas por un trabajo del tipo de experimento controlado, tesugiero el
libro de (Lawson, 2014).

Finalmente, te sugiero consideres también el punto de vista que propone
para \textbf{repensar el uso de la estadística} este otro autor
(\emph{Statistical {Rethinking} {\textbar} {A} {Bayesian} {Course} with
{Examples} in {R} and {Stan}}, 2020).

\phantomsection\label{refs}
\begin{CSLReferences}{1}{0}
Lecturas sugeridas

\bibitem[\citeproctext]{ref-lawson_design_2014}
Lawson, J. (2014). \emph{Design and {Analysis} of {Experiments} with
{R}}. CRC press. \url{https://bit.ly/3DR0ipN}

\bibitem[\citeproctext]{ref-pearl_libro_2020}
Pearl, J., \& Mackenzie, D. (2020). \emph{El libro del porqué}. Pasado y
Presente.
\url{https://www.marcialpons.es/media/pdf/ellibrodelporquejudeapearl.pdf}

\bibitem[\citeproctext]{ref-mcelreath_statistical_2020}
\emph{Statistical {Rethinking} {\textbar} {A} {Bayesian} {Course} with
{Examples} in {R} and {Stan}}. (2020). Chapman \& Hall.
\url{https://www.taylorfrancis.com/books/mono/10.1201/9781315372495/statistical-rethinking-richard-mcelreath}

\bibitem[\citeproctext]{ref-velasquez_exploring_nodate}
Velásquez, and I. C., Rebecca J. Powell. (s.~f.). \emph{Exploring
{Complex} {Survey} {Data} {Analysis} {Using} {R}}. Recuperado 14 de
marzo de 2025, de
\url{https://tidy-survey-r.github.io/tidy-survey-book/}

\end{CSLReferences}




\end{document}
